\documentclass[12pt]{article}
\usepackage[utf8]{inputenc}
\usepackage[T1]{fontenc}
\usepackage{amsmath}
\usepackage{amssymb}
\usepackage{xcolor}       % Voor kleuren
\usepackage{geometry}
\geometry{a4paper, margin=1in}

% Definieer kleuren (zelfde als vorige keer)
\definecolor{headerBrown}{RGB}{139,69,19}     % Bruin voor headers
\definecolor{vectorTeal}{RGB}{70, 160, 160}     % Zacht blauwgroen voor vectoren
\definecolor{scalarBrightBlue}{RGB}{0, 0, 255}   % Helder blauw voor scalairen
\definecolor{matrixSoftRed}{RGB}{230,  70, 70}  % Zacht rood voor matrices

% Commando voor vectoren (Teal)
\renewcommand{\vec}[1]{\textcolor{vectorTeal}{\mathbf{#1}}}

% Helper commando voor scalaire output (BrightBlue)
\newcommand{\scalar}[1]{\textcolor{scalarBrightBlue}{#1}}
% Helper commando voor matrices (SoftRed)
\newcommand{\mat}[1]{\textcolor{matrixSoftRed}{#1}}

% Commando voor punten (standaard zwart)
\newcommand{\punt}[1]{\mathrm{#1}}

\pagestyle{empty} % Verwijder paginanummering

\begin{document}

\begin{center}
\Large \textbf{Formularium Kwadrieken}
\end{center}

\vspace{1em} % Extra verticale ruimte

% Definities voor symbolen
Een kwadriek is een oppervlak met algemene vergelijking (t.o.v. een orthonormaal stelsel):
\[
\scalar{F(x,y,z)} = \scalar{a_{11}x^2+a_{22}y^2+a_{33}z^2+2a_{12}xy+2a_{13}xz+2a_{23}yz + 2a_{14}x+2a_{24}y+2a_{34}z+a_{44}=0}
\]
met minstens één coëfficiënt van een kwadratische term verschillend van nul.
In matrixvorm (homogene coördinaten $\mat{X} = [\scalar{x, y, z, 1}]^T$): $\mat{X^T A X = 0}$.
Na een geschikte coördinatentransformatie (translatie en rotatie) kan elke kwadriek (in de nieuwe coördinaten $\scalar{x', y', z'}$) beschreven worden door één van de volgende standaardvergelijkingen. We laten de accenten weg en nemen aan $\scalar{a, b, c, p, q > 0}$.

% --- Sectie: Standaardvergelijkingen van Kwadrieken ---
\vspace{1.5em} % Marge boven header
{\centering
\textcolor{headerBrown}{\large\textbf{Standaardvergelijkingen van Kwadrieken}}
\par
}%

\textit{Ellipsoïde:}
\[
\frac{\scalar{x^2}}{\scalar{a^2}} + \frac{\scalar{y^2}}{\scalar{b^2}} + \frac{\scalar{z^2}}{\scalar{c^2}} = \scalar{1}
\]
(Bol als $\scalar{a=b=c}$)

\vspace{1em}
\textit{Hyperboloïde met één blad:} (Omwentelingsas langs Z)
\[
\frac{\scalar{x^2}}{\scalar{a^2}} + \frac{\scalar{y^2}}{\scalar{b^2}} - \frac{\scalar{z^2}}{\scalar{c^2}} = \scalar{1}
\]
(Permutaties voor andere omwentelingsassen mogelijk)

\vspace{1em}
\textit{Hyperboloïde met twee bladen:} (As langs Z)
\[
-\frac{\scalar{x^2}}{\scalar{a^2}} - \frac{\scalar{y^2}}{\scalar{b^2}} + \frac{\scalar{z^2}}{\scalar{c^2}} = \scalar{1} \quad \left(\text{of } \frac{\scalar{x^2}}{\scalar{a^2}} + \frac{\scalar{y^2}}{\scalar{b^2}} - \frac{\scalar{z^2}}{\scalar{c^2}} = \scalar{-1}\right)
\]
(Permutaties voor andere assen mogelijk)

\vspace{1em}
\textit{Elliptische Kegel:} (As langs Z)
\[
\frac{\scalar{x^2}}{\scalar{a^2}} + \frac{\scalar{y^2}}{\scalar{b^2}} - \frac{\scalar{z^2}}{\scalar{c^2}} = \scalar{0}
\]
(Permutaties voor andere assen mogelijk)

\vspace{1em}
\textit{Elliptische Paraboloïde:} (Top in oorsprong, as langs Z, opent naar boven)
\[
\scalar{z} = \frac{\scalar{x^2}}{\scalar{p}} + \frac{\scalar{y^2}}{\scalar{q}}
\]
(Permutaties/richtingen mogelijk)

\vspace{1em}
\textit{Hyperbolische Paraboloïde (Zadeloppervlak):} (Top in oorsprong, as langs Z)
\[
\scalar{z} = \frac{\scalar{x^2}}{\scalar{p}} - \frac{\scalar{y^2}}{\scalar{q}}
\]
(Permutaties/richtingen mogelijk)

\vspace{1em}
\textit{Elliptische Cilinder:} (Beschrijvenden parallel aan Z-as)
\[
\frac{\scalar{x^2}}{\scalar{a^2}} + \frac{\scalar{y^2}}{\scalar{b^2}} = \scalar{1}
\]
(Permutaties voor andere asrichtingen mogelijk)

\vspace{1em}
\textit{Hyperbolische Cilinder:} (Beschrijvenden parallel aan Z-as)
\[
\frac{\scalar{x^2}}{\scalar{a^2}} - \frac{\scalar{y^2}}{\scalar{b^2}} = \scalar{1} \quad \text{of} \quad -\frac{\scalar{x^2}}{\scalar{a^2}} + \frac{\scalar{y^2}}{\scalar{b^2}} = \scalar{1}
\]
(Permutaties voor andere asrichtingen mogelijk)

\vspace{1em}
\textit{Parabolische Cilinder:} (Beschrijvenden parallel aan Z-as, top rakend aan XZ-vlak)
\[
\scalar{y^2 = 2px} \quad (\text{of } \scalar{x^2 = 2py})
\]
(Permutaties/richtingen mogelijk)

% --- Sectie: Ontaarde Kwadrieken ---
\vspace{1.5em} % Marge boven header
{\centering
\textcolor{headerBrown}{\large\textbf{Ontaarde Kwadrieken (Voorbeelden)}}
\par
}%
\textit{Twee snijdende vlakken:} $\frac{\scalar{x^2}}{\scalar{a^2}} - \frac{\scalar{y^2}}{\scalar{b^2}} = \scalar{0}$

\vspace{0.5em}
\textit{Twee evenwijdige vlakken:} $\scalar{x^2 = a^2}$

\vspace{0.5em}
\textit{Twee samenvallende vlakken:} $\scalar{x^2 = 0}$

\vspace{0.5em}
\textit{(Rechte) Lijn (Z-as):} $\frac{\scalar{x^2}}{\scalar{a^2}} + \frac{\scalar{y^2}}{\scalar{b^2}} = \scalar{0}$

\vspace{0.5em}
\textit{Punt (Oorsprong):} $\frac{\scalar{x^2}}{\scalar{a^2}} + \frac{\scalar{y^2}}{\scalar{b^2}} + \frac{\scalar{z^2}}{\scalar{c^2}} = \scalar{0}$

\vspace{0.5em}
\textit{Lege verzameling (Imaginair):} Bv. $\frac{\scalar{x^2}}{\scalar{a^2}} + \frac{\scalar{y^2}}{\scalar{b^2}} + \frac{\scalar{z^2}}{\scalar{c^2}} = \scalar{-1}$, of $\scalar{x^2 = -a^2}$, of $\frac{\scalar{x^2}}{\scalar{a^2}} + \frac{\scalar{y^2}}{\scalar{b^2}} = \scalar{-1}$


\end{document}