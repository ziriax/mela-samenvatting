\documentclass[12pt]{article}
\usepackage[utf8]{inputenc}
\usepackage[T1]{fontenc}
\usepackage{amsmath}
\usepackage{amssymb}
\usepackage{xcolor}       % Voor kleuren
\usepackage{geometry}
\geometry{a4paper, margin=1in}

% Definieer kleuren (zelfde als vorige keer)
\definecolor{headerBrown}{RGB}{139,69,19}     % Bruin voor headers
\definecolor{vectorTeal}{RGB}{70, 160, 160}     % Zacht blauwgroen voor vectoren
\definecolor{scalarBrightBlue}{RGB}{0, 0, 255}   % Helder blauw voor scalairen
\definecolor{matrixSoftRed}{RGB}{230,  70, 70}  % Zacht rood voor matrices

% Commando voor vectoren (Teal)
\renewcommand{\vec}[1]{\textcolor{vectorTeal}{\mathbf{#1}}}

% Helper commando voor scalaire output (BrightBlue)
\newcommand{\scalar}[1]{\textcolor{scalarBrightBlue}{#1}}
% Helper commando voor matrices (SoftRed) - Past kleur toe op de hele matrix omgeving
\newcommand{\mat}[1]{\textcolor{matrixSoftRed}{#1}}

% Commando voor punten (standaard zwart)
\newcommand{\punt}[1]{\mathrm{#1}}

\pagestyle{empty} % Verwijder paginanummering

\begin{document}

\begin{center}
\Large \textbf{Formularium Affiene Transformaties}
\end{center}

\vspace{1em} % Extra verticale ruimte

% Definities voor symbolen
Zij $\punt{P}(\scalar{x, y, z})$ een punt met plaatsvector $\vec{p}$, en beeld $\punt{P'}(\scalar{x', y', z'})$ met plaatsvector $\vec{p'}$ onder transformatie $\varphi$.
Zij $\mat{A}$ de matrix van de geassocieerde lineaire afbeelding $L$.
Zij $\vec{t}$ de beeldvector van de oorsprong $\vec{\varphi(O)}$.
Zij $\scalar{\theta}$ een rotatiehoek, $\scalar{s_x, s_y, s_z}$ schaalfactoren, $\scalar{k}$ een shearfactor.

% --- Sectie: Algemene Affiene Transformatie ---
\vspace{1.5em} % Marge boven header
{\centering
\textcolor{headerBrown}{\large\textbf{Algemene Affiene Transformatie}}
\par
}%
\textit{Definitie (Barycentrisch):} $\varphi(\sum \scalar{\alpha_i} \punt{P_i}) = \sum \scalar{\alpha_i} \varphi(\punt{P_i})$ voor $\sum \scalar{\alpha_i} = \scalar{1}$.

\textit{Vectorieel:}
\[
\vec{p'} = \vec{\varphi(O)} + L(\vec{p}) = \vec{t} + L(\vec{p})
\]
\[
\vec{\varphi(\punt{P})\varphi(\punt{Q})} = L(\vec{\punt{PQ}})
\]
\textit{Matrixvorm (t.o.v. oorsprong):}
\[
\vec{p'} = \vec{t} + \mat{A} \vec{p}
\]
\textit{Homogene Coördinaten (3D):}
\[
\begin{pmatrix} \scalar{x'} \\ \scalar{y'} \\ \scalar{z'} \\ \scalar{1} \end{pmatrix}
=
\mat{\begin{pmatrix}
\scalar{a_{11}} & \scalar{a_{12}} & \scalar{a_{13}} & \scalar{t_x} \\
\scalar{a_{21}} & \scalar{a_{22}} & \scalar{a_{23}} & \scalar{t_y} \\
\scalar{a_{31}} & \scalar{a_{32}} & \scalar{a_{33}} & \scalar{t_z} \\
\scalar{0} & \scalar{0} & \scalar{0} & \scalar{1}
\end{pmatrix}}
\begin{pmatrix} \scalar{x} \\ \scalar{y} \\ \scalar{z} \\ \scalar{1} \end{pmatrix}
\quad \text{of} \quad
\begin{pmatrix} \vec{p'} \\ \scalar{1} \end{pmatrix} = \mat{\begin{pmatrix} A & \vec{t} \\ \vec{0}^T & 1 \end{pmatrix}} \begin{pmatrix} \vec{p} \\ \scalar{1} \end{pmatrix}
\]
\textit{Homogene Coördinaten (2D):}
\[
\begin{pmatrix} \scalar{x'} \\ \scalar{y'} \\ \scalar{1} \end{pmatrix}
=
\mat{\begin{pmatrix}
\scalar{a_{11}} & \scalar{a_{12}} & \scalar{t_x} \\
\scalar{a_{21}} & \scalar{a_{22}} & \scalar{t_y} \\
\scalar{0} & \scalar{0} & \scalar{1}
\end{pmatrix}}
\begin{pmatrix} \scalar{x} \\ \scalar{y} \\ \scalar{1} \end{pmatrix}
\]

% --- Sectie: Isometrieën (Behouden Afstanden) ---
\vspace{1.5em} % Marge boven header
{\centering
\textcolor{headerBrown}{\large\textbf{Isometrieën (Behouden Afstanden)}}
\par
}%
\textit{Translatie ($\vec{u} = (\scalar{u_x, u_y, u_z})$):}
\[
\vec{p'} = \vec{p} + \vec{u}
\quad (\mat{A=I}, \vec{t}=\vec{u})
\]
\[
\text{Homogeen (3D): } \mat{T(\vec{u})} = \mat{\begin{pmatrix} \scalar{1} & \scalar{0} & \scalar{0} & \scalar{u_x} \\ \scalar{0} & \scalar{1} & \scalar{0} & \scalar{u_y} \\ \scalar{0} & \scalar{0} & \scalar{1} & \scalar{u_z} \\ \scalar{0} & \scalar{0} & \scalar{0} & \scalar{1} \end{pmatrix}}
\]
\textit{Rotatie (2D, rond O, hoek $\scalar{\theta}$):}
\[
\mat{R(\scalar{\theta})} = \mat{\begin{pmatrix} \scalar{\cos\theta} & \scalar{-\sin\theta} \\ \scalar{\sin\theta} & \scalar{\cos\theta} \end{pmatrix}}
\quad (\vec{t}=\vec{0})
\]
\[
\text{Homogeen (2D): } \mat{\begin{pmatrix} \scalar{\cos\theta} & \scalar{-\sin\theta} & \scalar{0} \\ \scalar{\sin\theta} & \scalar{\cos\theta} & \scalar{0} \\ \scalar{0} & \scalar{0} & \scalar{1} \end{pmatrix}}
\]
\textit{Rotaties (3D, rond assen, hoek $\scalar{\theta}$):}
\[
\mat{R_x(\scalar{\theta})} = \mat{\begin{pmatrix} \scalar{1} & \scalar{0} & \scalar{0} \\ \scalar{0} & \scalar{\cos\theta} & \scalar{-\sin\theta} \\ \scalar{0} & \scalar{\sin\theta} & \scalar{\cos\theta} \end{pmatrix}}
\quad
\mat{R_y(\scalar{\theta})} = \mat{\begin{pmatrix} \scalar{\cos\theta} & \scalar{0} & \scalar{\sin\theta} \\ \scalar{0} & \scalar{1} & \scalar{0} \\ \scalar{-\sin\theta} & \scalar{0} & \scalar{\cos\theta} \end{pmatrix}}
\quad
\mat{R_z(\scalar{\theta})} = \mat{\begin{pmatrix} \scalar{\cos\theta} & \scalar{-\sin\theta} & \scalar{0} \\ \scalar{\sin\theta} & \scalar{\cos\theta} & \scalar{0} \\ \scalar{0} & \scalar{0} & \scalar{1} \end{pmatrix}}
\]
\textit{Reflectie (2D, t.o.v. O):} $\mat{R_{O}} = \mat{\begin{pmatrix} \scalar{-1} & \scalar{0} \\ \scalar{0} & \scalar{-1} \end{pmatrix}}$
\textit{Reflectie (2D, t.o.v. assen):} $\mat{R_{x-as}} = \mat{\begin{pmatrix} \scalar{1} & \scalar{0} \\ \scalar{0} & \scalar{-1} \end{pmatrix}}$, $\mat{R_{y-as}} = \mat{\begin{pmatrix} \scalar{-1} & \scalar{0} \\ \scalar{0} & \scalar{1} \end{pmatrix}}$
\textit{Reflectie (3D, t.o.v. vlakken):} $\mat{R_{xy}} = \mat{\text{diag}(\scalar{1, 1, -1})}$, $\mat{R_{yz}} = \mat{\text{diag}(\scalar{-1, 1, 1})}$, $\mat{R_{xz}} = \mat{\text{diag}(\scalar{1, -1, 1})}$


% --- Sectie: Andere Affiene Transformaties ---
\vspace{1.5em} % Marge boven header
{\centering
\textcolor{headerBrown}{\large\textbf{Andere Affiene Transformaties}}
\par
}%
\textit{Schaling (rond O, factoren $\scalar{s_x, s_y, s_z}$):}
\[
\mat{S(\scalar{s_x, s_y, s_z})} = \mat{\begin{pmatrix} \scalar{s_x} & \scalar{0} & \scalar{0} \\ \scalar{0} & \scalar{s_y} & \scalar{0} \\ \scalar{0} & \scalar{0} & \scalar{s_z} \end{pmatrix}}
\quad (\vec{t}=\vec{0})
\]
\[
\text{Homogeen (3D): } \mat{\begin{pmatrix} \scalar{s_x} & \scalar{0} & \scalar{0} & \scalar{0} \\ \scalar{0} & \scalar{s_y} & \scalar{0} & \scalar{0} \\ \scalar{0} & \scalar{0} & \scalar{s_z} & \scalar{0} \\ \scalar{0} & \scalar{0} & \scalar{0} & \scalar{1} \end{pmatrix}}
\]
\textit{Afschuiving (Shear) (2D):}
\[
\text{X-shear: } \mat{Sh_x(\scalar{k})} = \mat{\begin{pmatrix} \scalar{1} & \scalar{k} \\ \scalar{0} & \scalar{1} \end{pmatrix}}
\quad
\text{Y-shear: } \mat{Sh_y(\scalar{k})} = \mat{\begin{pmatrix} \scalar{1} & \scalar{0} \\ \scalar{k} & \scalar{1} \end{pmatrix}}
\]

% --- Sectie: Samenstelling ---
\vspace{1.5em} % Marge boven header
{\centering
\textcolor{headerBrown}{\large\textbf{Samenstelling}}
\par
}%
Als $\varphi = \varphi_n \circ \dots \circ \varphi_2 \circ \varphi_1$, met $\varphi_i$ horend bij homogene matrix $\mat{M_i}$.
\[
\text{Homogene matrix van } \varphi: \quad \mat{M_{totaal}} = \mat{M_n} \dots \mat{M_2} \mat{M_1}
\]
(Let op de volgorde: transformatie 1 wordt eerst toegepast).

% --- Sectie: Coördinatentransformaties (Passief) ---
\vspace{1.5em} % Marge boven header
{\centering
\textcolor{headerBrown}{\large\textbf{Coördinatentransformaties (Passief)}}
\par
}%
Relatie tussen oude coördinaten $(x,y,z)$ en nieuwe $(x',y',z')$.
\textit{Translatie van assenstelsel (Nieuwe oorsprong O' is oude punt $\vec{t}$):}
\[
\vec{p} = \vec{p'} + \vec{t} \quad \text{of} \quad \vec{p'} = \vec{p} - \vec{t}
\]
\[
\begin{pmatrix} \scalar{x} \\ \scalar{y} \\ \scalar{z} \\ \scalar{1} \end{pmatrix} = \mat{T(\vec{t})} \begin{pmatrix} \scalar{x'} \\ \scalar{y'} \\ \scalar{z'} \\ \scalar{1} \end{pmatrix}
\quad \text{vs. Actief: }
\begin{pmatrix} \scalar{x'} \\ \scalar{y'} \\ \scalar{z'} \\ \scalar{1} \end{pmatrix} = \mat{T(\vec{t})} \begin{pmatrix} \scalar{x} \\ \scalar{y} \\ \scalar{z} \\ \scalar{1} \end{pmatrix}
\]
\textit{Rotatie van assenstelsel (Nieuwe assen t.o.v. oude via $\mat{R}$):}
\[
\vec{p} = \mat{R} \vec{p'} \quad \text{of} \quad \vec{p'} = \mat{R^{-1}} \vec{p} = \mat{R^T} \vec{p} \quad (\text{voor rotatiematrix } R)
\]
\[
\begin{pmatrix} \vec{p} \\ \scalar{1} \end{pmatrix} = \mat{\begin{pmatrix} R & \vec{0} \\ \vec{0}^T & 1 \end{pmatrix}} \begin{pmatrix} \vec{p'} \\ \scalar{1} \end{pmatrix}
\quad \text{vs. Actief: }
\begin{pmatrix} \vec{p'} \\ \scalar{1} \end{pmatrix} = \mat{\begin{pmatrix} R & \vec{0} \\ \vec{0}^T & 1 \end{pmatrix}} \begin{pmatrix} \vec{p} \\ \scalar{1} \end{pmatrix}
\]

\end{document}