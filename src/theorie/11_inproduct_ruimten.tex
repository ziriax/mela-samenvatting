\documentclass[12pt]{article}
\usepackage[utf8]{inputenc}
\usepackage[T1]{fontenc}
\usepackage{amsmath}
\usepackage{amssymb}
\usepackage{xcolor}       % Voor kleuren
\usepackage{geometry}
\geometry{a4paper, margin=1in}

% Definieer kleuren (aangepast SoftRed)
\definecolor{headerBrown}{RGB}{139,69,19}     % Bruin voor headers
\definecolor{vectorTeal}{RGB}{70, 160, 160}     % Zacht blauwgroen voor vectoren
\definecolor{scalarBrightBlue}{RGB}{0, 0, 255}   % Helder blauw voor scalairen
\definecolor{matrixSoftRed}{RGB}{230, 70, 70}    % Zacht Rood voor matrices/operatoren

% Commando voor vectoren (Teal)
\renewcommand{\vec}[1]{\textcolor{vectorTeal}{\mathbf{#1}}}

% Helper commando voor scalaire output (BrightBlue) - Ook voor resultaat inproduct/norm
\newcommand{\scalar}[1]{\textcolor{scalarBrightBlue}{#1}}
% Helper commando voor matrices/operatoren (SoftRed)
\newcommand{\mat}[1]{\textcolor{matrixSoftRed}{#1}}

% Velden (standaard zwart)
\newcommand{\field}[1]{\mathbb{#1}}
% Inproduct notatie
\newcommand{\innerprod}[2]{\scalar{\langle #1, #2 \rangle}}
% Norm notatie
\newcommand{\norm}[1]{\scalar{\| #1 \|}}
% Projectie operator
\DeclareMathOperator{\proj}{proj}

\pagestyle{empty} % Verwijder paginanummering

\begin{document}

\begin{center}
\Large \textbf{Formularium Inproductruimten}
\end{center}

\vspace{1em} % Extra verticale ruimte

% Definities voor symbolen
Zij $V$ een vectorruimte over $\field{K}$ (met $\field{K} = \field{R}$ of $\field{K} = \field{C}$).
Vectoren: $\vec{u}, \vec{v}, \vec{w} \in V$. Scalairen: $\scalar{\lambda, \mu} \in \field{K}$.
Inproduct: $\innerprod{\cdot}{\cdot}: V \times V \to \field{K}$.

% --- Sectie: Inproduct: Definitie & Eigenschappen ---
\vspace{1.5em} % Marge boven header
{\centering
\textcolor{headerBrown}{\large\textbf{Inproduct: Definitie \& Eigenschappen}}
\par
}%
Een functie $\innerprod{\cdot}{\cdot}$ is een inproduct als $\forall \vec{v}, \vec{v_1}, \vec{v_2}, \vec{w} \in V, \forall \scalar{\lambda} \in \field{K}$:
\begin{itemize}
    \item[(i)] $\innerprod{v}{v} \ge \scalar{0}$ \quad en \quad $\innerprod{v}{v} = \scalar{0} \iff \vec{v} = \vec{0}$ \quad (Positief definiet)
    \item[(ii)] $\innerprod{\scalar{\lambda}\vec{v_1} + \vec{v_2}}{w} = \scalar{\lambda}\innerprod{v_1}{w} + \innerprod{v_2}{w}$ \quad (Lineair in 1e argument)
    \item[(iii)] $\innerprod{v}{w} = \scalar{\overline{\innerprod{w}{v}}}$ \quad (Hermitisch / Conjugaat symmetrisch; Symmetrisch als $\field{K}=\field{R}$)
\end{itemize}
Gevolg (Sesquilineariteit):
\[
\innerprod{v}{\scalar{\lambda}\vec{w_1} + \vec{w_2}} = \scalar{\overline{\lambda}} \innerprod{v}{w_1} + \innerprod{v}{w_2}
\]
Standaard inproduct op $\field{K}^n$: $\innerprod{x}{y} = \sum_{i=1}^{\scalar{n}} \scalar{x_i \overline{y_i}}$ (in $\field{R}^n$: $\sum \scalar{x_i y_i}$).

% --- Sectie: Norm & Ongelijkheid van Cauchy-Schwarz ---
\vspace{1.5em} % Marge boven header
{\centering
\textcolor{headerBrown}{\large\textbf{Norm & Ongelijkheid van Cauchy-Schwarz}}
\par
}%
\textit{Norm (geïnduceerd door inproduct):}
\[
\norm{v} = \scalar{\sqrt{\innerprod{v}{v}}}
\]
\textit{Eigenschappen Norm:}
\begin{itemize}
    \item $\norm{v} \ge \scalar{0}$ en $\norm{v} = \scalar{0} \iff \vec{v} = \vec{0}$
    \item $\norm{\scalar{\lambda} \vec{v}} = \scalar{|\lambda|} \norm{v}$
    \item $\norm{v + w} \le \norm{v} + \norm{w}$ (Driehoeksongelijkheid)
\end{itemize}
\textit{Ongelijkheid van Cauchy-Schwarz:}
\[
\scalar{|\innerprod{v}{w}|} \le \norm{v} \norm{w}
\]
Gelijkheid geldt $\iff \vec{v}$ en $\vec{w}$ zijn lineair afhankelijk.

% --- Sectie: Orthogonaliteit & Pythagoras ---
\vspace{1.5em} % Marge boven header
{\centering
\textcolor{headerBrown}{\large\textbf{Orthogonaliteit & Pythagoras}}
\par
}%
\textit{Orthogonaliteit:} Vectoren $\vec{v}, \vec{w}$ zijn orthogonaal ($\vec{v} \perp \vec{w}$) als:
\[
\innerprod{v}{w} = \scalar{0}
\]
\textit{Stelling van Pythagoras:}
\[
\vec{v} \perp \vec{w} \implies \norm{v + w}^2 = \norm{v}^2 + \norm{w}^2
\]
(Geldt ook voor meer onderling orthogonale vectoren).

% --- Sectie: Orthonormale Bases ---
\vspace{1.5em} % Marge boven header
{\centering
\textcolor{headerBrown}{\large\textbf{Orthonormale Bases}} ($V$ eindig-dimensionaal)
\par
}%
\textit{Orthogonale Verzameling:} $S = \{\vec{u_1}, \dots, \vec{u_k}\}$ waarbij $\innerprod{u_i}{u_j} = \scalar{0}$ voor $\scalar{i \neq j}$.
\textit{Orthonormale Verzameling (ONV):} Orthogonaal en $\norm{u_i} = \scalar{1}$ voor alle $i$.
\[
\innerprod{u_i}{u_j} = \scalar{\delta_{ij}} \quad (\text{Kronecker delta})
\]
Een ONV is altijd lineair onafhankelijk.
\textit{Orthonormale Basis (ONB):} Een ONV die een basis vormt voor $V$.
\textit{Coördinaten t.o.v. ONB $B = \{\vec{e_1}, \dots, \vec{e_n}\}$:}
\[
\vec{v} = \sum_{i=1}^{\scalar{n}} \innerprod{v}{e_i} \vec{e_i}
\]
\textit{Inproduct in ONB-coördinaten:} Als $[\vec{v}]_B = (\scalar{\alpha_i})$, $[\vec{w}]_B = (\scalar{\beta_i})$, dan:
\[
\innerprod{v}{w} = \sum_{i=1}^{\scalar{n}} \scalar{\alpha_i \overline{\beta_i}}
\]
\textit{Identiteit van Parseval:}
\[
\norm{v}^2 = \innerprod{v}{v} = \sum_{i=1}^{\scalar{n}} \scalar{|\innerprod{v}{e_i}|^2}
\]

% --- Sectie: Gram-Schmidt Procedure ---
\vspace{1.5em} % Marge boven header
{\centering
\textcolor{headerBrown}{\large\textbf{Gram-Schmidt Procedure}}
\par
}%
Procedure om een basis $\{\vec{v_1}, \dots, \vec{v_n}\}$ om te zetten naar een orthogonale basis $\{\vec{u_1}, \dots, \vec{u_n}\}$ en een orthonormale basis $\{\vec{e_1}, \dots, \vec{e_n}\}$.
\begin{align*}
\vec{u_1} &= \vec{v_1} \\
\vec{u_2} &= \vec{v_2} - \frac{\innerprod{v_2}{u_1}}{\innerprod{u_1}{u_1}} \vec{u_1} \\
\vec{u_3} &= \vec{v_3} - \frac{\innerprod{v_3}{u_1}}{\innerprod{u_1}{u_1}} \vec{u_1} - \frac{\innerprod{v_3}{u_2}}{\innerprod{u_2}{u_2}} \vec{u_2} \\
&\vdots \\
\vec{u_k} &= \vec{v_k} - \sum_{j=1}^{\scalar{k-1}} \frac{\innerprod{v_k}{u_j}}{\norm{u_j}^2} \vec{u_j} \\
&\vdots \\
\vec{u_n} &= \vec{v_n} - \sum_{j=1}^{\scalar{n-1}} \frac{\innerprod{v_n}{u_j}}{\norm{u_j}^2} \vec{u_j}
\end{align*}
Orthonormale basis: $\vec{e_k} = \frac{\vec{u_k}}{\norm{u_k}}$.

% --- Sectie: Orthogonaal Complement & Projectie ---
\vspace{1.5em} % Marge boven header
{\centering
\textcolor{headerBrown}{\large\textbf{Orthogonaal Complement & Projectie}}
\par
}%
Zij $W$ een deelruimte van $V$.
\textit{Orthogonaal Complement:}
\[
W^{\perp} = \{ \vec{v} \in V \mid \innerprod{v}{w} = \scalar{0} \text{ voor alle } \vec{w} \in W \}
\]
$W^{\perp}$ is een deelruimte van $V$.
\textit{Directe Som Decompositie (V eindig-dim.):}
\[
V = W \oplus W^{\perp}
\]
\[
(\vec{v} \in V \implies \exists! \vec{w} \in W, \exists! \vec{w'} \in W^{\perp} : \vec{v} = \vec{w} + \vec{w'})
\]
\[
\scalar{\Dim(V)} = \scalar{\Dim(W)} + \scalar{\Dim(W^{\perp})}
\]
\[
(W^{\perp})^{\perp} = W
\]
\textit{Orthogonale Projectie op $W$:} Zij $\{\vec{e_1}, \dots, \vec{e_k}\}$ een ONB voor $W$.
\[
\mat{\proj}_W(\vec{v}) = \sum_{i=1}^{\scalar{k}} \innerprod{v}{e_i} \vec{e_i}
\]
De vector $\mat{\proj}_W(\vec{v})$ is de unieke vector in $W$ die de afstand $\norm{v - w}$ minimaliseert voor $\vec{w} \in W$.
De vector $\vec{v} - \mat{\proj}_W(\vec{v})$ ligt in $W^{\perp}$.

\end{document}