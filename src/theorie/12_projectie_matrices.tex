\documentclass[12pt]{article}
\usepackage[utf8]{inputenc}
\usepackage[T1]{fontenc}
\usepackage{amsmath}
\usepackage{amssymb}
\usepackage{xcolor}       % Voor kleuren
\usepackage{geometry}
\geometry{a4paper, margin=1in}

% Definieer kleuren (aangepast SoftRed)
\definecolor{headerBrown}{RGB}{139,69,19}     % Bruin voor headers
\definecolor{vectorTeal}{RGB}{70, 160, 160}     % Zacht blauwgroen voor vectoren
\definecolor{scalarBrightBlue}{RGB}{0, 0, 255}   % Helder blauw voor scalairen
\definecolor{matrixSoftRed}{RGB}{230, 70, 70}    % Zacht Rood voor matrices/operatoren

% Commando voor vectoren (Teal)
\renewcommand{\vec}[1]{\textcolor{vectorTeal}{\mathbf{#1}}}

% Helper commando voor scalaire output (BrightBlue)
\newcommand{\scalar}[1]{\textcolor{scalarBrightBlue}{#1}}
% Helper commando voor matrices/operatoren (SoftRed)
\newcommand{\mat}[1]{\textcolor{matrixSoftRed}{#1}}

% Identiteitsmatrix
\newcommand{\matI}{\mat{I}}
% Kolomruimte en Kern
\DeclareMathOperator{\Col}{Col}
\DeclareMathOperator{\Ker}{Ker}
\DeclareMathOperator{\Dim}{dim}

\pagestyle{empty} % Verwijder paginanummering

\begin{document}

\begin{center}
\Large \textbf{Formularium Orthogonale Projectiematrices}
\end{center}

\vspace{1em} % Extra verticale ruimte

% Definities voor symbolen
Zij $W$ een deelruimte van $\mathbb{R}^m$ (met standaard inproduct/dot product). De \textbf{orthogonale projectie} van $\vec{b} \in \mathbb{R}^m$ op $W$ is de unieke vector $\vec{p} \in W$ zodat $\vec{b} - \vec{p} \perp W$. De \textbf{orthogonale projectiematrix} $\mat{P}$ op $W$ is de $m \times m$ matrix waarvoor geldt $\vec{p} = \mat{P}\vec{b}$.

% --- Sectie: Definitie & Betekenis ---
\vspace{1.5em} % Marge boven header
{\centering
\textcolor{headerBrown}{\large\textbf{Definitie & Betekenis}}
\par
}%
De projectie $\vec{p}$ is de "beste benadering" van $\vec{b}$ binnen de deelruimte $W$. De vector $\vec{b} - \vec{p}$ staat loodrecht op \textit{elke} vector in $W$.

\textit{Waarom $\vec{p} = \mat{A}\vec{x}_{\text{ls}}$?}\\
Als $\{\vec{a_1}, \dots, \vec{a_n}\}$ een basis is voor $W$ (kolommen van matrix $\mat{A}$), dan moet de projectie $\vec{p}$ een lineaire combinatie zijn van deze basisvectoren: $\vec{p} = \scalar{x_1}\vec{a_1} + \dots + \scalar{x_n}\vec{a_n} = \mat{A}\vec{x}$ voor zekere coëfficiënten $\vec{x}$.
De eis dat $\vec{b} - \vec{p}$ loodrecht staat op $W$ betekent dat $\vec{b} - \mat{A}\vec{x}$ loodrecht staat op elke $\vec{a_j}$:
\[ \vec{a_j}^T (\vec{b} - \mat{A}\vec{x}) = \scalar{0} \quad \text{voor } j = 1, \dots, n \]
Dit kan in matrixvorm geschreven worden als:
\[ \mat{A}^T (\vec{b} - \mat{A}\vec{x}) = \vec{0} \quad \implies \quad \mat{A^T A} \vec{x} = \mat{A^T} \vec{b} \]
Dit zijn de \textit{normale vergelijkingen}. De unieke oplossing $\vec{x}_{\text{ls}}$ (als $\mat{A}$ lin. onafh. kolommen heeft) geeft de coëfficiënten die nodig zijn om de projectie $\vec{p} = \mat{A}\vec{x}_{\text{ls}}$ te vormen.

% --- Sectie: Constructie via (Willekeurige) Basis van W ---
\vspace{1.5em} % Marge boven header
{\centering
\textcolor{headerBrown}{\large\textbf{Constructie via (Willekeurige) Basis van W}}
\par
}%
Zij $\mat{A}$ een $m \times n$ matrix waarvan de kolommen een basis vormen voor $W$ (dus $\text{rank}(\mat{A}) = n$). De projectiematrix is:
\[
\mat{P} = \mat{A (\mat{A^T A})^{-1} A^T} \quad (m \times m \text{ matrix})
\]
De projectie van $\vec{b}$ op $W = \Col(\mat{A})$ is $\vec{p} = \mat{P}\vec{b}$.

% --- Sectie: Constructie via Orthonormale Basis van W ---
\vspace{1.5em} % Marge boven header
{\centering
\textcolor{headerBrown}{\large\textbf{Constructie via Orthonormale Basis van W}}
\par
}%
Zij $\{\vec{q_1}, \dots, \vec{q_n}\}$ een \textit{orthonormale} basis (ONB) voor $W$. Vorm de $m \times n$ matrix $\mat{Q}$ met deze vectoren als kolommen ($\mat{Q^T Q = I_n}$). De projectiematrix is:
\[
\mat{P} = \mat{Q Q^T} \quad (m \times m \text{ matrix})
\]
De projectie van $\vec{b}$ op $W = \Col(\mat{Q})$ is $\vec{p} = \mat{P}\vec{b} = \mat{Q Q^T} \vec{b} = \sum_{i=1}^{\scalar{n}} \scalar{(\vec{b} \cdot \vec{q_i})} \vec{q_i}$.

% --- Sectie: Eigenschappen van Projectiematrix P ---
\vspace{1.5em} % Marge boven header
{\centering
\textcolor{headerBrown}{\large\textbf{Eigenschappen van Projectiematrix} $\mat{P}$}
\par
}%
Zij $\mat{P}$ de orthogonale projectiematrix op de deelruimte $W$.
\begin{itemize}
    \item \textbf{Idempotent:} $\mat{P^2 = P}$
    \item \textbf{Symmetrisch:} $\mat{P^T = P}$
    \item \textbf{Beeld:} $\Col(\mat{P}) = W$
    \item \textbf{Kern:} $\Ker(\mat{P}) = W^{\perp}$
    \item \textbf{Effect op vectoren:} $\vec{w} \in W \implies \mat{P}\vec{w} = \vec{w}$; \quad $\vec{v} \in W^{\perp} \implies \mat{P}\vec{v} = \vec{0}$
    \item \textbf{Complementaire Projectie:} $\mat{I - P}$ is de projectiematrix op $W^{\perp}$.
\end{itemize}

% --- Sectie: Eigenwaarden en Multipliciteiten ---
\vspace{1.5em} % Marge boven header
{\centering
\textcolor{headerBrown}{\large\textbf{Eigenwaarden en Multipliciteiten}}
\par
}%
De enige mogelijke eigenwaarden van een projectiematrix $\mat{P}$ zijn $\scalar{1}$ en $\scalar{0}$.
\begin{itemize}
    \item \textbf{Eigenwaarde $\scalar{\lambda = 1}$:}
        \begin{itemize}
            \item Eigenruimte $E_1 = \{ \vec{x} \mid \mat{P}\vec{x} = \scalar{1}\vec{x} \} = W$.
            \item Meetkundige multipliciteit: $\scalar{\dim(E_1) = \dim(W)}$.
            \item Algebraïsche multipliciteit: Gelijk aan de meetkundige multipliciteit, dus $\scalar{\dim(W)}$. Dit is de rang van $\mat{P}$.
        \end{itemize}
    \item \textbf{Eigenwaarde $\scalar{\lambda = 0}$:}
        \begin{itemize}
            \item Eigenruimte $E_0 = \{ \vec{x} \mid \mat{P}\vec{x} = \scalar{0}\vec{x} \} = \Ker(\mat{P}) = W^{\perp}$.
            \item Meetkundige multipliciteit: $\scalar{\dim(E_0) = \dim(W^{\perp})}$.
            \item Algebraïsche multipliciteit: Gelijk aan de meetkundige multipliciteit, dus $\scalar{\dim(W^{\perp}) = m - \dim(W)}$. (Waarbij $\mat{P}$ een $m \times m$ matrix is).
        \end{itemize}
\end{itemize}
Aangezien $\scalar{\dim(W) + \dim(W^{\perp}) = m}$ (de dimensie van de totale ruimte $\mathbb{R}^m$), is de som van de algebraïsche multipliciteiten gelijk aan $m$, zoals verwacht voor een $m \times m$ matrix. Een projectiematrix is dus altijd diagonaliseerbaar.

\end{document}