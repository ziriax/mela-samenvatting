\documentclass[12pt]{article}
\usepackage[utf8]{inputenc}
\usepackage[T1]{fontenc}
\usepackage{amsmath}
\usepackage{amssymb}
\usepackage{xcolor}       % Voor kleuren
\usepackage{geometry}
\geometry{a4paper, margin=1in}

% Definieer kleuren (zelfde als vorige keer)
\definecolor{headerBrown}{RGB}{139,69,19}     % Bruin voor headers
\definecolor{vectorTeal}{RGB}{70, 160, 160}     % Zacht blauwgroen voor vectoren
\definecolor{scalarBrightBlue}{RGB}{0, 0, 255}   % Helder blauw voor scalairen
\definecolor{matrixSoftRed}{RGB}{230,  70, 70}  % Zacht rood voor matrices

% Commando voor vectoren (Teal)
\renewcommand{\vec}[1]{\textcolor{vectorTeal}{\mathbf{#1}}}

% Helper commando voor scalaire output (BrightBlue)
\newcommand{\scalar}[1]{\textcolor{scalarBrightBlue}{#1}}
% Helper commando voor matrices (SoftRed)
\newcommand{\mat}[1]{\textcolor{matrixSoftRed}{#1}}

% Commando voor punten (standaard zwart)
\newcommand{\punt}[1]{\mathrm{#1}}

% Velden (standaard zwart)
\newcommand{\field}[1]{\mathbb{#1}}

\pagestyle{empty} % Verwijder paginanummering

\begin{document}

\begin{center}
\Large \textbf{Formularium Lineaire Ruimten}
\end{center}

\vspace{1em} % Extra verticale ruimte

% Definities voor symbolen
Zij $V$ een verzameling (vectoren), $\field{F}$ een veld (scalairen, meestal $\field{R}$ of $\field{C}$).
Vectoren: $\vec{u}, \vec{v}, \vec{w} \in V$. Scalairen: $\scalar{\lambda, \mu, \alpha_i} \in \field{F}$.
Nulvector: $\vec{0} \in V$. Eenheidsscalair: $\scalar{1} \in \field{F}$.

% --- Sectie: Definitie Vectorruimte (Axioma's) ---
\vspace{1.5em} % Marge boven header
{\centering
\textcolor{headerBrown}{\large\textbf{Definitie Vectorruimte (Axioma's)}}
\par
}%
$(V, +)$ is een abelse groep:
\begin{itemize}
    \item $(\vec{u} + \vec{v}) + \vec{w} = \vec{u} + (\vec{v} + \vec{w})$ (Associativiteit +)
    \item $\vec{v} + \vec{0} = \vec{v}$ (Nulelement)
    \item $\forall \vec{v} \in V, \exists (-\vec{v}) \in V: \vec{v} + (-\vec{v}) = \vec{0}$ (Tegengestelde)
    \item $\vec{u} + \vec{v} = \vec{v} + \vec{u}$ (Commutativiteit +)
\end{itemize}
Scalaire vermenigvuldiging ($\field{F} \times V \to V: (\scalar{\lambda}, \vec{v}) \mapsto \scalar{\lambda}\vec{v}$):
\begin{itemize}
    \item $\scalar{\lambda}(\vec{u}+\vec{v}) = \scalar{\lambda}\vec{u} + \scalar{\lambda}\vec{v}$ (Distributiviteit 1)
    \item $(\scalar{\lambda+\mu})\vec{v} = \scalar{\lambda}\vec{v} + \scalar{\mu}\vec{v}$ (Distributiviteit 2)
    \item $\scalar{\lambda}(\scalar{\mu}\vec{v}) = \scalar{(\lambda\mu)}\vec{v}$ (Associativiteit mult.)
    \item $\scalar{1}\vec{v} = \vec{v}$ (Eenheidsscalair)
\end{itemize}

% --- Sectie: Lineaire Combinaties & Span ---
\vspace{1.5em} % Marge boven header
{\centering
\textcolor{headerBrown}{\large\textbf{Lineaire Combinaties \& Span}}
\par
}%
\textit{Lineaire Combinatie:} ($\vec{v_i} \in V, \scalar{\alpha_i} \in \field{F}$)
\[
\sum_{i=1}^{\scalar{k}} \scalar{\alpha_i} \vec{v_i} = \scalar{\alpha_1}\vec{v_1} + \scalar{\alpha_2}\vec{v_2} + \dots + \scalar{\alpha_k}\vec{v_k}
\]
\textit{Lineair Opspansel (Span):} ($S \subseteq V$)
\[
\text{span}(S) = \left\{ \sum_{i=1}^{\scalar{k}} \scalar{\alpha_i} \vec{v_i} \mid \scalar{k} \in \mathbb{N}, \vec{v_i} \in S, \scalar{\alpha_i} \in \field{F} \right\}
\]
($\text{span}(S)$ is de kleinste deelruimte van $V$ die $S$ bevat.)
\textit{Voortbrengende Verzameling:} $S$ brengt $V$ voort als $\text{span}(S) = V$.

% --- Sectie: Lineaire (On)afhankelijkheid ---
\vspace{1.5em} % Marge boven header
{\centering
\textcolor{headerBrown}{\large\textbf{Lineaire (On)afhankelijkheid}}
\par
}%
Een eindige verzameling $S = \{\vec{v_1}, \dots, \vec{v_k}\} \subseteq V$ is:
\textit{Lineair Onafhankelijk:}
\[
\sum_{i=1}^{\scalar{k}} \scalar{\alpha_i} \vec{v_i} = \vec{0} \quad \implies \quad \scalar{\alpha_1} = \scalar{\alpha_2} = \dots = \scalar{\alpha_k} = \scalar{0}
\]
(De nulvector kan enkel als de triviale lineaire combinatie geschreven worden.)

\textit{Lineair Afhankelijk:}
\[
\exists (\scalar{\alpha_1}, \dots, \scalar{\alpha_k}) \neq (\scalar{0}, \dots, \scalar{0}) \text{ zodat } \sum_{i=1}^{\scalar{k}} \scalar{\alpha_i} \vec{v_i} = \vec{0}
\]
(Minstens één vector is een lineaire combinatie van de andere.)

% --- Sectie: Basis & Dimensie ---
\vspace{1.5em} % Marge boven header
{\centering
\textcolor{headerBrown}{\large\textbf{Basis & Dimensie}}
\par
}%
\textit{Basis:} Een verzameling $B \subseteq V$ is een basis als:
\begin{itemize}
    \item $B$ is lineair onafhankelijk.
    \item $B$ is een voortbrengende verzameling voor $V$ ($\text{span}(B) = V$).
\end{itemize}
\textit{Unieke Voorstelling:} Als $B = \{\vec{e_1}, \dots, \vec{e_n}\}$ een basis is, dan kan elke $\vec{v} \in V$ uniek geschreven worden als:
\[
\vec{v} = \sum_{i=1}^{\scalar{n}} \scalar{\alpha_i} \vec{e_i}
\]
$(\scalar{\alpha_1}, \dots, \scalar{\alpha_n})$ zijn de coördinaten van $\vec{v}$ t.o.v. $B$.

\textit{Dimensie:} Als $V$ een eindige basis heeft, is het aantal vectoren in elke basis gelijk. Dit aantal is de dimensie van $V$.
\[
\scalar{\dim(V)} = \scalar{n} = |B|
\]

% --- Sectie: Deelruimten ---
\vspace{1.5em} % Marge boven header
{\centering
\textcolor{headerBrown}{\large\textbf{Deelruimten}}
\par
}%
Een niet-lege deelverzameling $W \subseteq V$ is een deelruimte als $W$ zelf een vectorruimte is over $\field{F}$ met de restricties van de bewerkingen in $V$. Equivalent:
\begin{itemize}
    \item $\vec{0} \in W$ (Impliciet als $W$ niet-leeg is en gesloten onder scalaire mult.)
    \item $\forall \vec{w_1}, \vec{w_2} \in W : \vec{w_1} + \vec{w_2} \in W$ (Gesloten onder +)
    \item $\forall \vec{w} \in W, \forall \scalar{\lambda} \in \field{F} : \scalar{\lambda}\vec{w} \in W$ (Gesloten onder scalaire mult.)
\end{itemize}
Equivalent Criterium:
\[
W \subseteq V \text{ is deelruimte} \iff \forall \vec{w_1}, \vec{w_2} \in W, \forall \scalar{\lambda, \mu} \in \field{F} : (\scalar{\lambda}\vec{w_1} + \scalar{\mu}\vec{w_2}) \in W
\]

% --- Sectie: Som & Directe Som ---
\vspace{1.5em} % Marge boven header
{\centering
\textcolor{headerBrown}{\large\textbf{Som & Directe Som}} ($V_1, V_2$ deelruimten van $V$)
\par
}%
\textit{Som:} Deelruimte voortgebracht door $V_1 \cup V_2$.
\[
V_1 + V_2 = \{ \vec{v_1} + \vec{v_2} \mid \vec{v_1} \in V_1, \vec{v_2} \in V_2 \}
\]
\textit{Doorsnede:} $V_1 \cap V_2$ is ook een deelruimte.

\textit{Dimensieformule van Grassmann:}
\[
\scalar{\dim(V_1 + V_2)} = \scalar{\dim(V_1)} + \scalar{\dim(V_2)} - \scalar{\dim(V_1 \cap V_2)}
\]
\textit{Directe Som:} De som $V_1 + V_2$ is direct als $V_1 \cap V_2 = \{\vec{0}\}$. Notatie: $V_1 \oplus V_2$.
\[
V = V_1 \oplus V_2 \iff \forall \vec{v} \in V, \exists! \vec{v_1} \in V_1, \exists! \vec{v_2} \in V_2 \text{ zodat } \vec{v} = \vec{v_1} + \vec{v_2}
\]
\[
V = V_1 \oplus V_2 \implies \scalar{\dim(V)} = \scalar{\dim(V_1)} + \scalar{\dim(V_2)}
\]

\end{document}