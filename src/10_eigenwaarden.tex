\documentclass[12pt]{article}
\usepackage[utf8]{inputenc}
\usepackage[T1]{fontenc}
\usepackage{amsmath}
\usepackage{amssymb}
\usepackage{xcolor}       % Voor kleuren
\usepackage{geometry}
\geometry{a4paper, margin=1in}

% Definieer kleuren (aangepast SoftRed)
\definecolor{headerBrown}{RGB}{139,69,19}     % Bruin voor headers
\definecolor{vectorTeal}{RGB}{70, 160, 160}     % Zacht blauwgroen voor vectoren
\definecolor{scalarBrightBlue}{RGB}{0, 0, 255}   % Helder blauw voor scalairen
\definecolor{matrixSoftRed}{RGB}{230, 70, 70}    % Zacht Rood voor matrices/transformaties

% Commando voor vectoren (Teal)
\renewcommand{\vec}[1]{\textcolor{vectorTeal}{\mathbf{#1}}}

% Helper commando voor scalaire output (BrightBlue)
\newcommand{\scalar}[1]{\textcolor{scalarBrightBlue}{#1}}
% Helper commando voor matrices en lineaire transformaties (SoftRed)
\newcommand{\mat}[1]{\textcolor{matrixSoftRed}{#1}}

% Operatoren voor Kern, Dim
\DeclareMathOperator{\Ker}{Ker}
\DeclareMathOperator{\Dim}{dim}
% Identiteits-transformatie/matrix
\newcommand{\Id}{\mat{\mathrm{Id}}}
\newcommand{\matI}{\mat{I}} % Identiteitsmatrix specifiek

\pagestyle{empty} % Verwijder paginanummering

\begin{document}

\begin{center}
\Large \textbf{Formularium Eigenwaarden en Eigenvectoren}
\end{center}

\vspace{1em} % Extra verticale ruimte

% Definities voor symbolen
Zij $V$ een vectorruimte over $\field{F}$. $\mat{T}: V \to V$ een lineaire transformatie (endomorfisme), of $\mat{A} \in \field{F}^{n \times n}$ een vierkante matrix.
$\scalar{\lambda} \in \field{F}$ is een scalair (eigenwaarde). $\vec{x} \in V$ of $\vec{x} \in \field{F}^{n \times 1}$ is een vector (eigenvector).

% --- Sectie: Definitie Eigenwaarde/Eigenvector ---
\vspace{1.5em} % Marge boven header
{\centering
\textcolor{headerBrown}{\large\textbf{Definitie Eigenwaarde/Eigenvector}}
\par
}%
Een scalair $\scalar{\lambda} \in \field{F}$ is een \textbf{eigenwaarde} van $\mat{T}$ (of $\mat{A}$) als er een \textit{niet-nul} vector $\vec{x} \in V$ (of $\field{F}^{n \times 1}$) bestaat, genaamd \textbf{eigenvector}, zodat:
\[
\mat{T}(\vec{x}) = \scalar{\lambda} \vec{x} \quad \text{(voor transformatie } \mat{T})
\]
\[
\mat{A} \vec{x} = \scalar{\lambda} \vec{x} \quad \text{(voor matrix } \mat{A})
\]
Equivalent herschreven:
\[
(\mat{T} - \scalar{\lambda} \Id_V) (\vec{x}) = \vec{0}
\]
\[
(\mat{A} - \scalar{\lambda} \matI) \vec{x} = \vec{0}
\]
($\Id_V$ is de identieke transformatie op $V$, $\matI$ is de $n \times n$ identiteitsmatrix).
Dit betekent dat $\scalar{\lambda}$ een eigenwaarde is als en slechts als de transformatie $\mat{T} - \scalar{\lambda} \Id_V$ (of de matrix $\mat{A} - \scalar{\lambda} \matI$) een \textit{niet-triviale kern} heeft.

% --- Sectie: Karakteristieke Vergelijking ---
\vspace{1.5em} % Marge boven header
{\centering
\textcolor{headerBrown}{\large\textbf{Karakteristieke Vergelijking}} (voor matrix $\mat{A} \in \field{F}^{n \times n}$)
\par
}%
De eigenwaarden $\scalar{\lambda}$ zijn de scalairen waarvoor $\mat{A} - \scalar{\lambda} \matI$ singulier is.
\[
\mat{\det(A - \lambda I)} = \scalar{0}
\]
Dit is de \textbf{karakteristieke vergelijking} van $\mat{A}$.
$\scalar{k_A(\lambda)} = \mat{\det(A - \lambda I)}$ is het \textbf{karakteristiek polynoom} van $\mat{A}$ (een polynoom van graad $\scalar{n}$ in $\scalar{\lambda}$).
De eigenwaarden zijn de wortels van het karakteristiek polynoom.

% --- Sectie: Eigenruimte ---
\vspace{1.5em} % Marge boven header
{\centering
\textcolor{headerBrown}{\large\textbf{Eigenruimte}}
\par
}%
Voor een eigenwaarde $\scalar{\lambda}$ van $\mat{T}$ (of $\mat{A}$) is de \textbf{eigenruimte} $E_\lambda$ de verzameling van alle eigenvectoren horend bij $\scalar{\lambda}$, aangevuld met de nulvector.
\[
E_\lambda = \{ \vec{x} \in V \mid \mat{T}(\vec{x}) = \scalar{\lambda} \vec{x} \} = \Ker(\mat{T} - \scalar{\lambda} \Id_V)
\]
\[
E_\lambda = \{ \vec{x} \in \field{F}^n \mid \mat{A} \vec{x} = \scalar{\lambda} \vec{x} \} = \Ker(\mat{A} - \scalar{\lambda} \matI)
\]
$E_\lambda$ is een deelruimte van $V$ (of $\field{F}^n$).
\textit{Meetkundige Multipliciteit:} $\scalar{r_\lambda} = \scalar{\Dim(E_\lambda)}$. Er geldt $\scalar{r_\lambda \ge 1}$.
\textit{Algebraïsche Multipliciteit:} $\scalar{q_\lambda}$ = multipliciteit van $\scalar{\lambda}$ als wortel van $\scalar{k_A(\lambda)=0}$.
\[
\scalar{1 \le r_\lambda \le q_\lambda}
\]

% --- Sectie: Diagonalisatie ---
\vspace{1.5em} % Marge boven header
{\centering
\textcolor{headerBrown}{\large\textbf{Diagonalisatie}} ($\mat{A} \in \field{F}^{n \times n}$)
\par
}%
Matrix $\mat{A}$ is \textbf{diagonaliseerbaar} over $\field{F}$ als er een inverteerbare matrix $\mat{P} \in \field{F}^{n \times n}$ bestaat zodat $\mat{P^{-1} A P = D}$, waarbij $\mat{D}$ een diagonaalmatrix is.
\textit{Equivalent Criterium 1:} $\mat{A}$ is diagonaliseerbaar $\iff$ er bestaat een basis van $\field{F}^n$ bestaande uit eigenvectoren van $\mat{A}$.
\textit{Equivalent Criterium 2:} $\mat{A}$ is diagonaliseerbaar $\iff$
\begin{itemize}
    \item Het karakteristiek polynoom $\scalar{k_A(\lambda)}$ ontbindt volledig in lineaire factoren over $\field{F}$.
    \item Voor elke eigenwaarde $\scalar{\lambda}$ geldt $\scalar{r_\lambda = q_\lambda}$ (meetkundige mult. = algebraïsche mult.).
\end{itemize}
\textit{Constructie:} Als $\mat{A}$ diagonaliseerbaar is, dan zijn de kolommen van $\mat{P}$ de $\scalar{n}$ lineair onafhankelijke eigenvectoren, en $\mat{D}$ is de diagonaalmatrix met de corresponderende eigenwaarden op de diagonaal (in dezelfde volgorde).
\[
\mat{A} = \mat{P D P^{-1}}
\]

% --- Sectie: Symmetrische Matrices (Reëel) ---
\vspace{1.5em} % Marge boven header
{\centering
\textcolor{headerBrown}{\large\textbf{Symmetrische Matrices (Reëel)}} ($\mat{A} \in \field{R}^{n \times n}, \mat{A^T = A}$)
\par
}%
\begin{itemize}
    \item Alle eigenwaarden van $\mat{A}$ zijn reëel ($\scalar{\lambda \in \field{R}}$).
    \item Eigenvectoren horend bij \textit{verschillende} eigenwaarden zijn orthogonaal.
    \item $\mat{A}$ is altijd \textbf{orthogonaal diagonaliseerbaar}: Er bestaat een \textit{orthogonale} matrix $\mat{P}$ ($\mat{P^T = P^{-1}}$) zodat $\mat{P^T A P = D}$ (diagonaal).
\end{itemize}
De kolommen van $\mat{P}$ vormen een orthonormale basis van $\field{R}^n$ bestaande uit eigenvectoren van $\mat{A}$.

\end{document}