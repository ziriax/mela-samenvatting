\documentclass[12pt]{article}
\usepackage[utf8]{inputenc}
\usepackage[T1]{fontenc}
\usepackage{amsmath}
\usepackage{amssymb}
\usepackage{xcolor}       % Voor kleuren
\usepackage{geometry}
\geometry{a4paper, margin=1in}

% Definieer kleuren (zelfde als vorige keer)
\definecolor{headerBrown}{RGB}{139,69,19}     % Bruin voor headers
\definecolor{vectorTeal}{RGB}{70, 160, 160}     % Zacht blauwgroen voor vectoren
\definecolor{scalarBrightBlue}{RGB}{0, 0, 255}   % Helder blauw voor scalairen
\definecolor{matrixSoftRed}{RGB}{230, 80, 80}  % Zacht rood voor matrices

% Commando voor vectoren (Teal) - hier gebruikt voor kolomvectoren x, b
\renewcommand{\vec}[1]{\textcolor{vectorTeal}{\mathbf{#1}}}

% Helper commando voor scalaire output (BrightBlue)
\newcommand{\scalar}[1]{\textcolor{scalarBrightBlue}{#1}}
% Helper commando voor matrices (SoftRed)
\newcommand{\mat}[1]{\textcolor{matrixSoftRed}{#1}}

% Commando voor punten (standaard zwart)
\newcommand{\punt}[1]{\mathrm{#1}}
% Commando voor rang (Rank)
\DeclareMathOperator{\rk}{rk}

\pagestyle{empty} % Verwijder paginanummering

\begin{document}

\begin{center}
\Large \textbf{Formularium Matrices, Determinanten en Stelsels}
\end{center}

\vspace{1em} % Extra verticale ruimte

% Definities voor symbolen
Zij $\mat{A} = (\scalar{a_{ij}}), \mat{B} = (\scalar{b_{ij}})$ matrices, $\scalar{\lambda}$ een scalair.
$\mat{I}$ is de identiteitsmatrix, $\mat{O}$ de nulmatrix.
$\vec{x}$ is een kolomvector van onbekenden, $\vec{b}$ een kolomvector van constanten.

% --- Sectie: Matrixbewerkingen ---
\vspace{1.5em} % Marge boven header
{\centering
\textcolor{headerBrown}{\large\textbf{Matrixbewerkingen}}
\par
}%
\textit{Som ($\mat{A}, \mat{B}$ van zelfde dimensie $m \times n$):}
\[
\mat{A} + \mat{B} = (\scalar{a_{ij} + b_{ij}})
\]
\textit{Scalaire Vermenigvuldiging:}
\[
\scalar{\lambda} \mat{A} = (\scalar{\lambda a_{ij}})
\]
\textit{Matrix Vermenigvuldiging ($\mat{A}$ is $m \times n$, $\mat{B}$ is $n \times p$):} $\mat{C} = \mat{A}\mat{B}$ is $m \times p$ met
\[
\scalar{c_{ik}} = \sum_{j=1}^{\scalar{n}} \scalar{a_{ij} b_{jk}}
\]
\textit{Eigenschappen Product:} $\mat{A}(\mat{BC}) = (\mat{AB})\mat{C}$, $\mat{A}(\mat{B+C}) = \mat{AB} + \mat{AC}$, $(\mat{A+B})\mat{C} = \mat{AC} + \mat{BC}$. Let op: $\mat{AB} \neq \mat{BA}$ in het algemeen.

\textit{Transponeren:} $\mat{A^T} = (\scalar{a_{ji}})$ (rijen worden kolommen).
\[
\mat{(A+B)^T} = \mat{A^T + B^T}, \quad \mat{(\lambda A)^T} = \scalar{\lambda} \mat{A^T}, \quad \mat{(AB)^T} = \mat{B^T A^T}, \quad \mat{(A^T)^T} = \mat{A}
\]

% --- Sectie: Speciale Matrices ---
\vspace{1.5em} % Marge boven header
{\centering
\textcolor{headerBrown}{\large\textbf{Speciale Matrices}} ($\mat{A}$ is vierkant)
\par
}%
\textit{Identiteitsmatrix $\mat{I}$:} $\mat{I} = (\scalar{\delta_{ij}})$ ($\delta_{ij} = 1$ als $i=j$, $0$ anders). $\mat{AI} = \mat{IA} = \mat{A}$.
\textit{Symmetrisch:} $\mat{A^T = A}$.
\textit{Anti-/Scheefsymmetrisch:} $\mat{A^T = -A}$.
\textit{Orthogonaal:} $\mat{A^T A} = \mat{A A^T} = \mat{I}$ (equivalent: $\mat{A^T = A^{-1}}$). Kolommen/rijen vormen orthonormale basis.

% --- Sectie: Determinanten ---
\vspace{1.5em} % Marge boven header
{\centering
\textcolor{headerBrown}{\large\textbf{Determinanten}} ($\mat{A}$ is $n \times n$)
\par
}%
\textit{Definitie (Co-factorontwikkeling naar rij $i$):} $\mat{C_{ij}} = \scalar{(-1)^{i+j}} \mat{M_{ij}}$ (cofactor), $\mat{M_{ij}}$ (minor) is determinant van submatrix na schrappen rij $i$, kolom $j$.
\[
\mat{\det(A)} = \sum_{j=1}^{\scalar{n}} \scalar{a_{ij}} \mat{C_{ij}} = \sum_{j=1}^{\scalar{n}} \scalar{(-1)^{i+j} a_{ij}} \mat{M_{ij}}
\]
(Ook ontwikkeling naar kolom $j$ mogelijk: $\mat{\det(A)} = \sum_{i=1}^{\scalar{n}} \scalar{a_{ij}} \mat{C_{ij}}$)

\textit{Geval 2x2:} $\mat{\det \begin{pmatrix} \scalar{a} & \scalar{b} \\ \scalar{c} & \scalar{d} \end{pmatrix}} = \scalar{ad - bc}$

\textit{Eigenschappen:}
\[
\mat{\det(A^T)} = \mat{\det(A)}
\]
\[
\mat{\det(AB)} = \mat{\det(A)}\mat{\det(B)}
\]
\[
\mat{\det(I)} = \scalar{1}
\]
Rij/kolom operaties: Wisselen $\implies \times \scalar{(-1)}$, $\times \scalar{\lambda} \implies \times \scalar{\lambda}$, Combinatie $\implies$ onveranderd.
\[
\mat{\det(A)} = \scalar{0} \iff \mat{A} \text{ is singulier (niet-inverteerbaar)} \iff \text{rijen/kolommen zijn lineair afhankelijk.}
\]

% --- Sectie: Inverse Matrix ---
\vspace{1.5em} % Marge boven header
{\centering
\textcolor{headerBrown}{\large\textbf{Inverse Matrix}} ($\mat{A}$ is $n \times n$)
\par
}%
\textit{Definitie:} $\mat{A^{-1}}$ is de matrix waarvoor $\mat{A A^{-1}} = \mat{A^{-1} A} = \mat{I}$.
\textit{Bestaan:} $\mat{A^{-1}}$ bestaat $\iff \mat{\det(A)} \neq \scalar{0}$.
\textit{Formule (Adjunctmatrix):} $\mat{\text{adj}(A)} = \mat{C^T}$ (getransponeerde van cofactormatrix $\mat{C} = (\mat{C_{ij}})$).
\[
\mat{A^{-1}} = \frac{\scalar{1}}{\mat{\det(A)}} \mat{\text{adj}(A)} = \frac{\scalar{1}}{\mat{\det(A)}} \mat{C^T}
\]
\textit{Eigenschappen:}
\[
\mat{(A^{-1})^{-1}} = \mat{A}, \quad \mat{(AB)^{-1}} = \mat{B^{-1} A^{-1}}, \quad \mat{(A^T)^{-1}} = \mat{(A^{-1})^T}
\]

% --- Sectie: Stelsels Lineaire Vergelijkingen ---
\vspace{1.5em} % Marge boven header
{\centering
\textcolor{headerBrown}{\large\textbf{Stelsels Lineaire Vergelijkingen}}
\par
}%
Stelsel van $m$ vergelijkingen in $n$ onbekenden: $\mat{A} \vec{x} = \vec{b}$. $\mat{A}$ is $m \times n$, $\vec{x}$ is $n \times 1$, $\vec{b}$ is $m \times 1$.
Uitgebreide matrix: $\mat{[A | \vec{b}]}$.
Rang: $\scalar{\rk(A)}$ = max. aantal lin. onafh. rijen/kolommen.

\textit{Stelling van Rouché-Capelli (Hoofdwet):}
\[
\text{Stelsel } \mat{A} \vec{x} = \vec{b} \text{ heeft oplossing(en)} \iff \scalar{\rk(A)} = \scalar{\rk([A | \vec{b}])}
\]
\textit{Aantal Oplossingen (als oplossing bestaat, $\scalar{r = \rk(A)}$):}
\begin{itemize}
    \item Unieke oplossing $\iff \scalar{r = n}$ (aantal onbekenden).
    \item Oneindig veel oplossingen $\iff \scalar{r < n}$. Dimensie oplossingsruimte = $\scalar{n - r}$. Algemene oplossing = particuliere oplossing + algemene oplossing homogeen stelsel.
\end{itemize}
\textit{Homogeen Stelsel ($\mat{A} \vec{x} = \vec{0}$):} Altijd triviale oplossing $\vec{x} = \vec{0}$.
\[
\text{Niet-triviale oplossingen} \iff \scalar{\rk(A) < n} \quad (\text{Voor vierkante matrix } \mat{A}: \iff \mat{\det(A)} = \scalar{0})
\]
\textit{Vierkant Stelsel ($\mat{A}$ is $n \times n$):}
\[
\text{Unieke oplossing} \iff \mat{\det(A)} \neq \scalar{0}. \quad \text{Oplossing: } \vec{x} = \mat{A^{-1}} \vec{b}
\]
\textit{Regel van Cramer (als $\mat{\det(A)} \neq \scalar{0}$):} $\mat{A_i}$ is $\mat{A}$ met $i$-de kolom vervangen door $\vec{b}$.
\[
\scalar{x_i} = \frac{\mat{\det(A_i)}}{\mat{\det(A)}}
\]
\textit{Methode van Gauss(-Jordan):} Gebruik elementaire rijoperaties op $\mat{[A|\vec{b}]}$ om (gereduceerde) rij-echelonvorm te bekomen en stelsel op te lossen.

\end{document}